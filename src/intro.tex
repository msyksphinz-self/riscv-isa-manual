\begin{comment}
\chapter{Introduction}
\end{comment}
\chapter{イントロダクション}

\begin{comment}
RISC-V (pronounced ``risk-five'') is a new instruction-set
architecture (ISA) that was originally designed to support computer
architecture research and education, but which we now hope will also
become a standard free and open architecture for industry
implementations.  Our goals in defining RISC-V include:
\end{comment}
RISC-V(``リスクファイブ''と発音)は、コンピュータアーキテクチャの研究と
教育をサポートするために設計された新しい命令セットアーキテクチャ(ISA)ですが、
現在私たちが望んでいるのは、業界標準の自由でオープンなアーキテクチャです。
RISC-Vを定義する私たちの目標は次のとおりです:

\begin{comment}
\vspace{-0.1in}
\begin{itemize}
\parskip 0pt
\itemsep 1pt
\item A completely {\em open} ISA that is freely available to
  academia and industry.
\item A {\em real} ISA suitable for direct native hardware implementation,
  not just simulation or binary translation.
\item An ISA that avoids ``over-architecting'' for a particular
  microarchitecture style (e.g., microcoded, in-order, decoupled,
  out-of-order) or implementation technology (e.g., full-custom, ASIC,
  FPGA), but which allows efficient implementation in any of these.
\item An ISA separated into a {\em small} base integer ISA, usable by
  itself as a base for customized accelerators or for educational
  purposes, and optional standard extensions, to support
  general-purpose software development.
\item Support for the revised 2008 IEEE-754 floating-point standard~\cite{ieee754-2008}.
\item An ISA supporting extensive ISA extensions and
  specialized variants.
\item Both 32-bit and 64-bit address space variants for
  applications, operating system kernels, and hardware implementations.
\item An ISA with support for highly-parallel multicore
  or manycore implementations, including heterogeneous multiprocessors.
\item Optional {\em variable-length instructions} to both expand available
  instruction encoding space and to support an optional {\em dense
  instruction encoding} for improved performance, static code size,
  and energy efficiency.
\item A fully virtualizable ISA to ease hypervisor development.
\item An ISA that simplifies experiments with new privileged architecture designs.
\end{itemize}
\vspace{-0.1in}
\end{comment}

\vspace{-0.1in}
\begin{itemize}
\parskip 0pt
\itemsep 1pt
\item 学界や産業界が自由に利用できる完全に{\em オープンな}ISAです。
\item 実際のISAシミュレーションまたはバイナリ変換だけでなく、直接ネイティブハードウェア実装に適しています。
\item ISAは特定のマイクロアーキテクチャスタイル(例えばマイクロコード化、インオーダー、デカップル、アウトオブオーダー)や
  実装技術(フルカスタム、ASIC、FPGAなど)を``オーバーアーキテクチャー``することなく、
  効率的にこれらのいずれかの実装を可能にします。
\item ISAは、汎用のソフトウェア開発をサポートするために、
  カスタマイズされたアクセラレータまたは教育目的のためのベースとして使用可能な{\em 小さな}基本整数ISAと
  オプションの標準拡張子に分かれています。
\item 改訂された2008 IEEE-754浮動小数点標準のサポート [14]。
\item 広範なユーザーレベルのISA拡張機能と{\em 特定用途向けの}バリアントをサポートするISA。
\item アプリケーション、オペレーティングシステムカーネル、およびハードウェア実装の32ビットおよび64ビットアドレス空間の両バリエーション。
\item 異種マルチプロセッサを含む、高度に並列なマルチコアまたは多くのコアの実装を支援するISA。
\item オプションの可変長命令は、使用可能な命令エンコーディング空間を拡張と、
  オプションの高密度命令エンコーディングを提供し、パフォーマンス、
  静的コードサイズ、およびエネルギー効率を向上させます。
\item ハイパーバイザー開発を容易にする完全仮想化ISA
\item 新しいスーパーバイザーレベルとハイパーバイザーレベルのISAデザインを使用して実験を簡単にするISA。
\end{itemize}
\vspace{-0.1in}

\begin{comment}
\begin{commentary}
  Commentary on our design decisions is formatted as in this
  paragraph.  This non-normative text can be skipped if the reader is
  only interested in the specification itself.
\end{commentary}

\begin{commentary}
  私たちのデザイン決定に関する論評(解説)は、
  この段落のように書式化されており、
  読者が仕様書そのもののみに興味がある場合スキップすることができます。
\end{commentary}

\begin{comment}
\begin{commentary}
The name RISC-V was chosen to represent the fifth major RISC ISA
design from UC Berkeley (RISC-I~\cite{riscI-isca1981},
RISC-II~\cite{Katevenis:1983}, SOAR~\cite{Ungar:1984}, and
SPUR~\cite{spur-jsscc1989} were the first four).  We also pun on the
use of the Roman numeral ``V'' to signify ``variations'' and
``vectors'', as support for a range of architecture research,
including various data-parallel accelerators, is an explicit goal of
the ISA design.
\end{commentary}
\end{comment}

\begin{commentary}
  RISC-Vという名前は、UC Berkeley(RISC-I~\cite{riscI-isca1981},
    RISC-II~\cite{Katevenis:1983}、SOAR~\cite{Ungar:1984}、
    SPUR~\cite{spur-jsscc1989}の最初の4つ)から5番目の主要なRISC
  ISA設計を代表するものです。
  さまざまなデータ並列アクセラレータを含む一連のアーキテクチャ研究の支援がISA設計の明白な目標で、
  ``バリエーション``と ``ベクトル``を表すローマ数字``V``の使用についてもしゃれ
  (だじゃれ)を言います。
\end{commentary}

RISC-V ISAはソフトウェアからのインタフェースのみを可視化し、
特定のハードウェアだけでなく様々な実装に対応できるよう、
実装の詳細について可能な限り触れないように定義されています(しかしコメント文については、
実装依存の決定についても含まれています)。
RISC-Vマニュアルは2巻で構成されています。
本巻はベースとなる{\em 非特権}命令のデザインについてカバーしており、
オプションとなる非特権ISA拡張も含まれています。
非特権命令は通常すべての特権モードと特権アーキテクチャで使用可能なものですが、
命令の動作は特権モードと特権アーキテクチャに非常に強く依存します。
第2巻は、最初の(``クラシックな``)特権アーキテクチャのデザインについて説明します。
このマニュアルはIEC 80000-13:2008で定義される単位系を使用しており、1バイトは8ビットとします。

\begin{comment}
The RISC-V ISA is defined avoiding implementation details as much as
possible (although commentary is included on implementation-driven
decisions) and should be read as the software-visible interface to a
wide variety of implementations rather than as the design of a
particular hardware artifact.  The RISC-V manual is structured in two
volumes.  This volume covers the design of the base {\em unprivileged}
instructions, including optional unprivileged ISA extensions.
Unprivileged instructions are those that are generally usable in all
privilege modes in all privileged architectures, though behavior might
vary depending on privilege mode and privilege architecture.  The
second volume provides the design of the first (``classic'')
privileged architecture. The manuals use IEC 80000-13:2008
conventions, with a byte of 8 bits.
\end{comment}

\begin{comment}
\begin{commentary}
In the unprivileged ISA design, we tried to remove any dependence on
particular microarchitectural features, such as cache line size, or on
privileged architecture details, such as page translation.  This is
both for simplicity and to allow maximum flexibility for alternative
microarchitectures or alternative privileged architectures.
\end{commentary}
\end{comment}

非特権ISAのデザインでは、キャッシュラインサイズなどの特定のマイクロアーキテクチャの機能や、
ページ変換などの特権アーキテクチャの詳細に依存しないようにしました。 
これは、単純化するためであり、また、代替のマイクロアーキテクチャや
代替の特権アーキテクチャに対して最大限の柔軟性を持たせるためでもあります。

\begin{comment}
\section{RISC-V Hardware Platform Terminology}

A RISC-V hardware platform can contain one or more RISC-V-compatible
processing cores together with other non-RISC-V-compatible cores,
fixed-function accelerators, various physical memory structures, I/O
devices, and an interconnect structure to allow the components to
communicate.

A component is termed a {\em core} if it contains an independent
instruction fetch unit.  A RISC-V-compatible core might support
multiple RISC-V-compatible hardware threads, or {\em harts}, through
multithreading.

A RISC-V core might have additional specialized instruction-set
extensions or an added {\em coprocessor}.  We use the term {\em
  coprocessor} to refer to a unit that is attached to a RISC-V core
and is mostly sequenced by a RISC-V instruction stream, but which
contains additional architectural state and instruction-set
extensions, and possibly some limited autonomy relative to the
primary RISC-V instruction stream.

We use the term {\em accelerator} to refer to either a
non-programmable fixed-function unit or a core that can operate
autonomously but is specialized for certain tasks.  In RISC-V systems,
we expect many programmable accelerators will be RISC-V-based cores
with specialized instruction-set extensions and/or customized
coprocessors.  An important class of RISC-V accelerators are I/O
accelerators, which offload I/O processing tasks from the main
application cores.

The system-level organization of a RISC-V hardware platform can range
from a single-core microcontroller to a many-thousand-node cluster of
shared-memory manycore server nodes.  Even small systems-on-a-chip
might be structured as a hierarchy of multicomputers and/or
multiprocessors to modularize development effort or to provide secure
isolation between subsystems.
\end{comment}

\section{RISC-V ハードウェアプラットフォームの用語}

RISC-Vハードウェアプラットフォームには、1つまたは複数のRISC-V互換のプロセッシングコアと、
その他の非RISC-V互換のコア、固定機能のアクセラレータ、さまざまな物理メモリ構造、
I/Oデバイス、およびコンポーネント間の通信を可能にするインターコネクト構造が含まれます。

独立した命令フェッチユニットを含むコンポーネントを{\em コア}と呼びます。
RISC-V互換のコアは、マルチスレッドにより、複数のRISC-V互換のハードウェアスレッド、または{\em hart}をサポートします。

RISC-Vコアは、特殊な命令セットの拡張や、{\em コプロセッサ}の追加を行うことができます。
私たちは{\em コプロセッサ}という用語を、RISC-Vコアに接続されるユニットとして使用し、
ほとんどがRISC-V命令ストリームによって順番に実行されますが、
追加のアーキテクチャ状態や命令セット拡張を含み、
場合によってはプライマリRISC-V命令ストリームに対して限定的な自律性を持つユニットを指すために使用します。

ここでは、プログラマブルではない固定機能ユニットや、
自律的に動作するが特定のタスクに特化したコアを{\em アクセラレータ}と呼んでいます。 
RISC-Vシステムでは、プログラム可能なアクセラレータの多くは、RISC-Vベースのコアに特殊な
命令セット拡張やカスタマイズされたコプロセッサを搭載したものになると考えられます。
RISC-Vアクセラレータの中で重要なクラスは、
I/Oアクセラレータで、I/O処理タスクをメインのアプリケーションコアからオフロードします。

RISC-Vハードウェアプラットフォームのシステムレベルの構成は、
シングルコアのマイクロコントローラから、共有メモリを持つメニーコアの
サーバーノードの何千何万ノードのクラスタにまで及びます。 
また、小型のシステム・オン・チップであっても、マルチコンピュータやマルチプロセッサの
階層構造にすることで、開発作業をモジュール化したり、サブシステム間を安全に分離することができます。

\begin{comment}
\section{RISC-V Software Execution Environments and Harts}

The behavior of a RISC-V program depends on the execution environment
in which it runs.  A RISC-V execution environment interface (EEI)
defines the initial state of the program, the number and type of harts
in the environment including the privilege modes supported by the
harts, the accessibility and attributes of memory and I/O regions, the
behavior of all legal instructions executed on each hart (i.e., the
ISA is one component of the EEI), and the handling of any interrupts
or exceptions raised during execution including environment calls.
Examples of EEIs include the Linux application binary interface (ABI),
or the RISC-V supervisor binary interface (SBI).  The implementation
of a RISC-V execution environment can be pure hardware, pure software,
or a combination of hardware and software.  For example, opcode traps
and software emulation can be used to implement functionality not
provided in hardware.  Examples of execution environment
implementations include:
\begin{itemize}
  \item ``Bare metal'' hardware platforms where harts are directly
    implemented by physical processor threads and instructions have
    full access to the physical address space.  The hardware platform
    defines an execution environment that begins at power-on reset.
  \item RISC-V operating systems that provide multiple user-level
    execution environments by multiplexing user-level harts onto
    available physical processor threads and by controlling access to
    memory via virtual memory.
  \item RISC-V hypervisors that provide multiple supervisor-level
    execution environments for guest operating systems.
  \item RISC-V emulators, such as Spike, QEMU or rv8, which emulate
    RISC-V harts on an underlying x86 system, and which can provide
    either a user-level or a supervisor-level execution environment.
\end{itemize}

\begin{commentary}
  A bare hardware platform can be considered to define an EEI, where
  the accessible harts, memory, and other devices populate the
  environment, and the initial state is that at power-on reset.
  Generally, most software is designed to use a more abstract
  interface to the hardware, as more abstract EEIs provide greater
  portability across different hardware platforms.  Often EEIs are
  layered on top of one another, where one higher-level EEI uses
  another lower-level EEI.
\end{commentary}

From the perspective of software running in a given execution
environment, a hart is a resource that autonomously fetches and
executes RISC-V instructions within that execution environment.  In
this respect, a hart behaves like a hardware thread resource even if
time-multiplexed onto real hardware by the execution environment.
Some EEIs support the creation and destruction of additional harts,
for example, via environment calls to fork new harts.

The execution environment is responsible for ensuring the eventual forward
progress of each of its harts.
For a given hart, that responsibility is suspended while the hart is
exercising a mechanism that explicitly waits for an event, such as the
wait-for-interrupt instruction defined in Volume II of this specification; and
that responsibility ends if the hart is terminated.
The following events constitute forward progress:
\vspace{-0.2in}
\begin{itemize}
\parskip 0pt
\itemsep 1pt
\item The retirement of an instruction.
\item A trap, as defined in Section~\ref{sec:trap-defn}.
\item Any other event defined by an extension to constitute forward progress.
\end{itemize}

\begin{commentary}
The term hart was introduced in the work on
Lithe~\cite{lithe-pan-hotpar09,lithe-pan-pldi10} to provide a term to
represent an abstract execution resource as opposed to a software
thread programming abstraction.

The important distinction between a hardware thread (hart) and a
software thread context is that the software running inside an
execution environment is not responsible for causing progress of each
of its harts; that is the responsibility of the outer execution
environment.  So the environment's harts operate like hardware threads
from the perspective of the software inside the execution environment.

An execution environment implementation might time-multiplex a set of
guest harts onto fewer host harts provided by its own execution
environment but must do so in a way that guest harts operate like
independent hardware threads.  In particular, if there are more guest
harts than host harts then the execution environment must be able to
preempt the guest harts and must not wait indefinitely for guest
software on a guest hart to ``yield" control of the guest hart.
\end{commentary}
\end{comment}

\section{RISC-V ソフトウェア実行環境とhart}

RISC-Vプログラムの動作は、それが実行される実行環境に依存します。
RISC-Vの実行環境インターフェース(execution environment interface: EEI)は、
プログラムの初期状態、環境内のhartの数や種類(hartがサポートする特権モードを含む)、
メモリやI/O領域のアクセスや属性、
各hartで実行されるすべての合法的な命令の動作(すなわち、ISAはEEIの構成要素のひとつ)、
Environment Callを含む実行中に発生する割り込みや例外の処理などを定義します。
EEIの例としては、Linuxのアプリケーション・バイナリ・インターフェース(ABI)やRISC-Vのスーパーバイザ・バイナリ・インターフェース(SBI)などがある。
RISC-Vの実行環境の実装は、純粋なハードウェア、純粋なソフトウェア、またはハードウェアとソフトウェアの組み合わせとなります。
例えば、ハードウェアで提供されない機能を実装するために、オペコードトラップやソフトウェアエミュレーションを使用することができます。
実行環境の実装の例としては、以下のものがあります:
\begin{itemize}
\item ``ベアメタル'' ハードウェア・プラットフォームとは、
物理的なプロセッサ・スレッドによってハードウェアが直接実装され、
命令が物理的なアドレス空間に完全にアクセスできるハードウェア・プラットフォームのことです。
ードウェアプラットフォームは、パワーオンリセット時に始まる実行環境を定義します。
\item ユーザーレベルの命令を物理的なプロセッサスレッドに多重化したり、
仮想メモリを介してメモリへのアクセスを制御したりすることで、
複数のユーザーレベルの実行環境を提供するRISC-Vのオペレーティングシステムです。
\item ゲストOSに複数のスーパーバイザレベルの実行環境を提供する RISC-V ハイパーバイザ。
\item Spike, QEMU, rv8 などの RISC-V エミュレータは、x86 システム上の RISC-V hartをエミュレートし、ユーザレベルまたはスーパバイザレベルの実行環境を提供します。
\end{itemize}

\begin{commentary}
ベアメタルハードウェアプラットフォームは、アクセス可能なハードウェア、
メモリ、およびその他のデバイスが環境を構成し、
パワーオンリセット時の初期状態がEEIを定義していると考えることができます。
一般的に、ほとんどのソフトウェアは、ハードウェアに対してより抽象的な
インターフェースを使用するように設計されています。
これは、より抽象的なEEIは、異なるハードウェアプラットフォーム間で
より高い移植性を提供するからです。
多くの場合、EEIは上位のEEIが下位のEEIを使用するように、互いにレイヤされています。
\end{commentary}

ある実行環境で動作するソフトウェアの観点から見ると、
hartはその実行環境内でRISC-V命令を自律的にフェッチして実行するリソースです。 
この点では、実行環境によって実ハードウェアに時間多重化されていても、
hartはハードウェアのスレッドリソースのように振る舞います。
EEIの中には、Environment Callで新しいHartをフォークするなど、
追加のHartの生成と破棄をサポートするものもあります。

実行環境は、各hartの最終的な進行を保証する責任があります。
あるhartについて、本仕様書の第2巻で定義されているwait-for-interrupt命令のように、
hartが明示的にイベントを待つメカニズムを実行している間は、
その責任は中断され、hartが終了するとその責任は終了します。
以下のイベントが前進を構成します:

\vspace{-0.2in}
\begin{itemize}
\parskip 0pt
\itemsep 1pt
\item 命令のリタイア。
\item ~\ref{sec:trap-defn}節で定義されるトラップ。
\item 拡張で定義される前進させる他の任意のイベント。
\end{itemize}

\begin{commentary}
hartという用語は、Lithe~\cite{lithe-pan-hotpar09,lithe-pan-pldi10}の研究で導入されたもので、
ソフトウェア・スレッド・プログラミングの抽象化とは対照的に、
抽象的な実行リソースを表す用語として使われています。

ハードウェア・スレッド(hart)とソフトウェア・スレッド・コンテキストの重要な違いは、実行環境の内部で実行されているソフトウェアは、それぞれのhartの進行を引き起こす責任はなく、それは外部の実行環境の責任であるということです。 そのため、実行環境のhartは、実行環境内のソフトウェアの観点からは、
ハードウェア・スレッドのように動作します。

実行環境の実装は、自分の実行環境が提供するより少ないホストhartに、
一連のゲストhartを時間多重化するかもしれませんが、
ゲストhartが独立したハードウェアスレッドのように動作するようにしなければなりません。
特に、ホストhartよりも多くのゲストhartがある場合、
実行環境はゲストhartを先取りすることができなければならず、
ゲストhart上のゲストソフトウェアがゲストhartの制御を``ゆずる"のをいつまでも待っていてはいけません。

\end{commentary}

\begin{comment}
\section{RISC-V ISA Overview}

A RISC-V ISA is defined as a base integer ISA, which must be present
in any implementation, plus optional extensions to the base ISA.  The
base integer ISAs are very similar to that of the early RISC processors
except with no branch delay slots and with support for optional
variable-length instruction encodings.  A base is carefully
restricted to a minimal set of instructions sufficient to provide a
reasonable target for compilers, assemblers, linkers, and operating
systems (with additional privileged operations), and so provides
a convenient ISA and software toolchain ``skeleton'' around which more
customized processor ISAs can be built.

Although it is convenient to speak of {\em the} RISC-V ISA, RISC-V is
actually a family of related ISAs, of which there are currently four
base ISAs.  Each base integer instruction set is characterized by the
width of the integer registers and the corresponding size of the
address space and by the number of integer registers.  There are two
primary base integer variants, RV32I and RV64I, described in
Chapters~\ref{rv32} and \ref{rv64}, which provide 32-bit or 64-bit
address spaces respectively.  We use the term XLEN to refer to the
width of an integer register in bits (either 32 or 64).
Chapter~\ref{rv32e} describes the RV32E subset variant of the RV32I
base instruction set, which has been added to support small
microcontrollers, and which has half the number of integer registers.
Chapter~\ref{rv128} sketches a future RV128I variant of the base
integer instruction set supporting a flat 128-bit address space
(XLEN=128).  The base integer instruction sets use a two's-complement
representation for signed integer values.

\begin{commentary}
Although 64-bit address spaces are a requirement for larger systems,
we believe 32-bit address spaces will remain adequate for many
embedded and client devices for decades to come and will be desirable
to lower memory traffic and energy consumption.  In addition, 32-bit
address spaces are sufficient for educational purposes.  A larger flat
128-bit address space might eventually be required, so we ensured this
could be accommodated within the RISC-V ISA framework.
\end{commentary}

\begin{commentary}
The four base ISAs in RISC-V are treated as distinct base ISAs.  A
common question is why is there not a single ISA, and in particular,
why is RV32I not a strict subset of RV64I?  Some earlier ISA designs
(SPARC, MIPS) adopted a strict superset policy when increasing address
space size to support running existing 32-bit binaries on new 64-bit
hardware.

The main advantage of explicitly separating base ISAs is that each
base ISA can be optimized for its needs without requiring to support
all the operations needed for other base ISAs.  For example, RV64I can
omit instructions and CSRs that are only needed to cope with the
narrower registers in RV32I.  The RV32I variants can use encoding
space otherwise reserved for instructions only required by wider
address-space variants.

The main disadvantage of not treating the design as a single ISA is
that it complicates the hardware needed to emulate one base ISA on
another (e.g., RV32I on RV64I).  However, differences in addressing
and illegal instruction traps generally mean some mode switch would be
required in hardware in any case even with full superset instruction
encodings, and the different RISC-V base ISAs are similar enough that
supporting multiple versions is relatively low cost.  Although some
have proposed that the strict superset design would allow legacy
32-bit libraries to be linked with 64-bit code, this is impractical in
practice, even with compatible encodings, due to the differences in
software calling conventions and system-call interfaces.

The RISC-V privileged architecture provides fields in {\tt
  misa} to control the unprivileged ISA at each level to support emulating
different base ISAs on the same hardware.  We note that newer SPARC
and MIPS ISA revisions have deprecated support for running 32-bit code
unchanged on 64-bit systems.

A related question is why there is a different encoding for 32-bit
adds in RV32I (ADD) and RV64I (ADDW)? The ADDW opcode could be used
for 32-bit adds in RV32I and ADDD for 64-bit adds in RV64I, instead of
the existing design which uses the same opcode ADD for 32-bit adds in
RV32I and 64-bit adds in RV64I with a different opcode ADDW for 32-bit
adds in RV64I.  This would also be more consistent with the use of the
same LW opcode for 32-bit load in both RV32I and RV64I.  The very
first versions of RISC-V ISA did have a variant of this alternate
design, but the RISC-V design was changed to the current choice in
January 2011.  Our focus was on supporting 32-bit integers in the
64-bit ISA not on providing compatibility with the 32-bit ISA, and the
motivation was to remove the asymmetry that arose from having not all
opcodes in RV32I have a *W suffix (e.g., ADDW, but AND not ANDW).  In
hindsight, this was perhaps not well-justified and a consequence of
designing both ISAs at the same time as opposed to adding one later to
sit on top of another, and also from a belief we had to fold platform
requirements into the ISA spec which would imply that all the RV32I
instructions would have been required in RV64I.  It is too late to
change the encoding now, but this is also of little practical
consequence for the reasons stated above.

It has been noted we could enable the *W variants as an extension to
RV32I systems to provide a common encoding across RV64I and a future
RV32 variant.
\end{commentary}

RISC-V has been designed to support extensive customization and
specialization.  Each base integer ISA can be extended with one or
more optional instruction-set extensions.  An extension may be
categorized as either standard, custom, or non-conforming.
For this purpose, we divide each RISC-V
instruction-set encoding space (and related encoding spaces such as
the CSRs) into three disjoint categories: {\em standard}, {\em
  reserved}, and {\em custom}.  Standard extensions and encodings
are defined by the Foundation; any extensions not defined by the
Foundation are {\em non-standard}.
Each base ISA and its standard extensions use only standard encodings,
and shall not conflict with each other in their uses of these encodings.
Reserved encodings are currently not defined but are saved for future
standard extensions; once thus used, they become standard encodings.
Custom encodings shall never be used for standard extensions and are
made available for vendor-specific non-standard extensions.
Non-standard extensions are either custom extensions, that use only
custom encodings, or {\em non-conforming} extensions, that use any
standard or reserved encoding.
Instruction-set extensions are generally shared but may provide slightly different
functionality depending on the base ISA.  Chapter~\ref{extensions}
describes various ways of extending the RISC-V ISA.  We have also
developed a naming convention for RISC-V base instructions and
instruction-set extensions, described in detail in
Chapter~\ref{naming}.

To support more general software development, a set of standard
extensions are defined to provide integer multiply/divide, atomic
operations, and single and double-precision floating-point arithmetic.
The base integer ISA is named ``I'' (prefixed by RV32 or RV64
depending on integer register width), and contains integer
computational instructions, integer loads, integer stores, and
control-flow instructions.  The standard integer multiplication and
division extension is named ``M'', and adds instructions to multiply
and divide values held in the integer registers.  The standard atomic
instruction extension, denoted by ``A'', adds instructions that
atomically read, modify, and write memory for inter-processor
synchronization.  The standard single-precision floating-point
extension, denoted by ``F'', adds floating-point registers,
single-precision computational instructions, and single-precision
loads and stores.  The standard double-precision floating-point
extension, denoted by ``D'', expands the floating-point registers, and
adds double-precision computational instructions, loads, and stores.
The standard ``C'' compressed instruction extension
provides narrower 16-bit forms of common instructions.

Beyond the base integer ISA and the standard GC extensions, we believe
it is rare that a new instruction will provide a significant benefit
for all applications, although it may be very beneficial for a certain
domain.  As energy efficiency concerns are forcing greater
specialization, we believe it is important to simplify the required
portion of an ISA specification.  Whereas other architectures usually
treat their ISA as a single entity, which changes to a new version as
instructions are added over time, RISC-V will endeavor to keep the
base and each standard extension constant over time, and instead layer
new instructions as further optional extensions.  For example, the
base integer ISAs will continue as fully supported standalone ISAs,
regardless of any subsequent extensions.
\end{comment}

\section{RISC-V ISAの概要}

RISC-V ISAは、すべての実装に存在しなければならない基本整数ISAと、
基本ISAへのオプションの拡張として定義されます。
基本整数ISAは、分岐遅延スロットがなく、オプションの可変長命令エンコーディングをサポートしている点を除いて、
初期のRISCプロセッサのものと非常によく似ています。
ベースはコンパイラ、アセンブラ、リンカ、およびオペレーティングシステム(追加の特権操作を含む)のための
合理的なターゲットを提供するのに十分な最小限の命令セットに注意深く制限されているので、
便利なISAおよびソフトウェアツールチェーン``スケルトン"よりカスタマイズされた
プロセッサISAを構築することができます。

RISC-V ISAという言い方は便利ですが、RISC-Vは実際には関連するISAのファミリーであり、
現在4つのベースISAがあります。 
各基本整数命令セットは、整数レジスタの幅とそれに対応するアドレス空間のサイズ、
および整数レジスタの数によって特徴付けられます。 
ベース整数にはRV32IとRV64Iの2種類があり、
第~\ref{rv32}章および第\ref{rv64}章で説明されており、
それぞれ32ビットと64ビットのアドレス空間を提供しています。
ここでは,整数レジスタの幅(32ビットまたは64ビット)をXLENと呼ぶことにします。
第~\ref{rv32e}章では、RV32I基本命令セットのサブセットであるRV32Eについて説明しています。
RV32Eは、小型マイクロコントローラをサポートするために追加された命令セットで、整数レジスタの数が半分になっています。
第~\ref{rv128}章では、フラットな128ビットのアドレス空間(XLEN=128)をサポートする基本整数命令セットの
将来のRV128Iバリアントをスケッチしています。
基本的な整数命令セットは、符号付き整数値に2の補数表現を使用します。

\begin{commentary}
大規模システムでは64ビットのアドレス空間が必要ですが、
32ビットのアドレス空間は多くのエンベデッド・デバイスやクライアント・デバイスにとって今後も数十年の間
十分なままであり、メモリ・トラフィックとエネルギー消費を削減することが望まれます。
さらに、教育目的では32ビットのアドレス空間で十分です。
将来的には128ビットのアドレス空間が必要になりますので、
これをRISC-V ISAフレームワークに収めることができます。
\end{commentary}

\begin{commentary}
RISC-Vの4つのベースISAは、それぞれ異なるベースISAとして扱われています。
よくある質問として、なぜ単一のISAではないのか、特にRV32IはなぜRV64Iの厳密なサブセットではないのか、
というものがあります。 
初期のISA設計(SPARC、MIPS)では、既存の32ビットバイナリを新しい64ビットハードウェアで実行するためにアドレス空間のサイズを大きくする際に、
厳密なスーパーセットポリシーを採用していました。

ベースISAを明示的に分離することの主な利点は、他のベースISAに必要なすべての操作をサポートする必要なく、
各ベースISAをそのニーズに合わせて最適化できることです。
例えば、RV64Iでは、RV32Iの狭いレジスタに対応するためだけに必要な命令やCSRを省略することができます。
RV32Iバリアントは、より広いアドレス空間のバリアントでのみ必要とされる命令のために確保されていた
エンコーディングスペースを使用することができます。

デザインを単一のISAとして扱わないことの主な欠点は、
あるベースISAを別のベースISAでエミュレートするために必要なハードウェアが複雑になることである(例えば、RV32IをRV64Iでエミュレートする場合)。
しかし、アドレッシングや不正命令トラップの違いから、完全なスーパーセット命令のエンコーディングであっても、
いずれにしてもハードウェアで何らかのモード切り替えが必要となるのが一般的であり、また、
異なるRISC-VのベースISAは十分に類似しているため、
複数のバージョンをサポートすることは比較的低コストです。
厳密なスーパーセット設計により、32ビットのレガシーライブラリを64ビットのコードとリンクさせることができるという提案もありましたが、
ソフトウェアの呼び出し規則やシステムコールのインターフェースが異なるため、
互換性のあるエンコーディングであっても実際には不可能です。

RISC-Vの特権アーキテクチャでは、各レベルの非特権ISAを制御するためのフィールドが{\tt misa}に用意されており、
同一のハードウェア上で異なるベースISAのエミュレーションをサポートしています。 
なお、SPARCやMIPSの新しいISAリビジョンでは、
64ビットシステム上で32ビットコードを変更せずに実行するサポートは廃止されています。

関連した質問として、RV32I(ADD)とRV64I(ADDW)の32ビット加算のエンコーディングが異なるのはなぜか、ということです。
従来のデザインでは、RV32Iの32ビット加算とRV64Iの64ビット加算に同じオペコードADDを使用し、
RV64Iの32ビット加算には異なるオペコードADDWを使用していましたが、
ADDWのオペコードをRV32Iの32ビット加算に使用し、RV64Iの64ビット加算にはADDDを使用することができます。
これは、RV32IとRV64Iの両方で32ビットのロードに同じLWオペコードを使用していることと、より一致します。
RISC-V ISAの最初のバージョンには、この代替デザインのバリエーションがありましたが、RISC-Vのデザインは2011年1月に現在の選択に変更されました。
これは、32ビットISAとの互換性ではなく、64ビットISAで32ビット整数をサポートすることに重点を置いたもので、
RV32Iのすべてのオペコードに*Wの接尾辞が付いていないことによる非対称性を解消することが動機となりました
(例:ADDWはあるが、ANDはANDWではない)。 
今にして思えば、これは、後からISAを追加するのではなく、両方のISAを同時に設計した結果であり、
また、プラットフォームの要求をISAの仕様に盛り込まなければならないという考えから、
RV32Iのすべての命令がRV64Iでも要求されることを意味していたため、十分に正当化されなかったのかもしれません。
今さらエンコーディングを変更することはできませんが、これも上記の理由から実際にはあまり意味がありません。

RV32Iシステムの拡張機能として*Wバリエーションを有効にして、
RV64Iと将来のRV32バリアントで共通のエンコーディングを提供することができると指摘されています。
\end{commentary}

\begin{comment}
\section{Memory}

A RISC-V hart has a single byte-addressable address space
of $2^{\text{XLEN}}$ bytes for all memory
accesses.  A {\em word} of memory is defined as \wunits{32}{bits}
(\wunits{4}{bytes}).  Correspondingly, a {\em halfword} is \wunits{16}{bits}
(\wunits{2}{bytes}), a {\em doubleword} is \wunits{64}{bits}
(\wunits{8}{bytes}), and a {\em quadword} is \wunits{128}{bits}
(\wunits{16}{bytes}).
The memory address space is circular, so that the byte at address
$2^{\text{XLEN}}-1$ is adjacent to the byte at address zero.  Accordingly, memory
address computations done by the hardware ignore overflow and instead
wrap around modulo $2^{\text{XLEN}}$.


The execution environment determines the mapping of hardware resources into
a hart's address space.
Different address ranges of a hart's address space may (1)~be vacant, or
(2)~contain {\em main memory}, or (3)~contain one or more {\em I/O devices}.
Reads and writes of I/O devices may have visible side effects, but accesses
to main memory cannot.
Although it is possible for the execution environment to call everything in
a hart's address space an I/O device, it is usually expected that some
portion will be specified as main memory.

When a RISC-V platform has multiple harts, the address spaces of any two
harts may be entirely the same, or entirely different, or may be partly
different but sharing some subset of resources, mapped into the same or
different address ranges.

\begin{commentary}
For a purely ``bare metal'' environment, all harts may see an identical
address space, accessed entirely by physical addresses.
However, when the execution environment includes an operating system
employing address translation, it is common for each hart to be given a
virtual address space that is largely or entirely its own.
\end{commentary}

Executing each RISC-V machine instruction entails one or more memory
accesses, subdivided into {\em
implicit} and {\em explicit} accesses.  For each instruction executed, an {\em
implicit} memory read (instruction fetch) is done to obtain the encoded
instruction to execute.  Many RISC-V instructions perform no further memory
accesses beyond instruction fetch.  Specific load and store instructions
perform an {\em explicit} read or write of memory at an address determined by
the instruction.  The execution environment may dictate that instruction
execution performs other {\em implicit} memory accesses (such as to implement
address translation) beyond those documented for the unprivileged ISA.

The execution environment determines what portions of the
non-vacant address space are
accessible for each kind of memory access.  For example, the set of locations
that can be implicitly read for instruction fetch may or may not have any
overlap with the set of locations that can be explicitly read by a load
instruction; and the set of locations that can be explicitly written by
a store instruction may be only a subset of locations that can be read.
Ordinarily, if an instruction attempts to access memory at an inaccessible
address, an exception is raised for the instruction.
Vacant locations in the address space are never accessible.

Except when specified otherwise, implicit reads that do not raise an
exception and that have no side effects
may occur arbitrarily early and speculatively, even before the machine could
possibly prove that the read will be needed.  For instance, a valid
implementation could attempt to read all of main memory at the earliest
opportunity, cache as many fetchable (executable) bytes as possible for later
instruction fetches, and avoid reading main memory for instruction fetches ever
again.  To ensure that certain implicit reads are ordered only after writes to
the same memory locations, software must execute specific fence or cache-control
instructions defined for this purpose (such as the FENCE.I instruction
defined in Chapter~\ref{chap:zifencei}).

The memory accesses (implicit or explicit) made by a hart may appear to occur
in a different order as perceived by another hart or by any other agent that
can access the same memory.  This perceived reordering of memory accesses is
always constrained, however, by the applicable memory consistency model.  The
default memory consistency model for RISC-V is the RISC-V Weak Memory Ordering
(RVWMO), defined in Chapter~\ref{ch:memorymodel} and in appendices.
Optionally, an implementation may adopt the stronger model of Total Store
Ordering, as defined in Chapter~\ref{sec:ztso}.  The execution environment may
also add constraints that further limit the perceived reordering of memory
accesses.
Since the RVWMO model is the weakest model allowed for any RISC-V
implementation, software written for this model is compatible with the
actual memory consistency rules of all RISC-V implementations.  As with
implicit reads, software must execute fence or cache-control instructions to
ensure specific ordering of memory accesses beyond the requirements of the
assumed memory consistency model and execution environment.
\end{comment}

\section{メモリ}

RISC-Vのhartは、すべてのメモリアクセスに対して、
$2^{text{XLEN}}$バイトのシングルバイトアドレッシング可能なアドレス空間を持っています。 
メモリの1{\em ワード}は、\wunits{32}{ビット}(\wunits{4}{バイト})と定義されます。
これに対応して、{\em ハーフワード}は\wunits{16}{ビット}(\wunits{2}{バイト})であり、
{\em ダブルワード}は\wunits{64}{ビット}(\wunits{8}{バイト})であり、
{\em クワッドワード}は\wunits{128}{ビット}(\wunits{16}{バイト})です。
また、メモリアドレス空間はリング状で,アドレス$2^{¥text{XLEN}}-1$のバイトはアドレス0のバイトに隣接しています。
そのため、ハードウェアによるメモリアドレスの計算では、オーバーフローを無視して、
modulo $2^{\text{XLEN}}$に折り返します。

実行環境によって、ハードウェアリソースのhartのアドレス空間へのマッピングが決まります。
hartのアドレス空間の異なるアドレスレンジは、(1)~空き領域、(2)~定義済み{\em メインメモリ}、(3)~1つ以上の定義済み{\em I/Oデバイス}のいずれかとなります。
I/Oデバイスの読み書きには目に見える副作用がありますが、
メインメモリへのアクセスには副作用はありません。
実行環境では、Hartのアドレス空間のすべてをI/Oデバイスと呼ぶことも可能ですが、
通常は一部をメインメモリとして指定することが想定されます。

RISC-Vプラットフォームに複数のhartsがある場合、
2つのhartsのアドレス空間は、全く同じであるか、全く異なるものであるか、
部分的に異なるが、同じまたは異なるアドレス範囲にマッピングされたリソースの
サブセットを共有するものであるかのいずれかです。

\begin{commentary}
純粋なベアメタル環境では、すべてのhartが同一のアドレス空間を持ち、
完全に物理アドレスでアクセスされます。
しかし、実行環境にアドレス変換機能を持つOSが含まれている場合、
各hartには、大部分または全体が独自の仮想アドレス空間が与えられるのが一般的です。
\end{commentary}

RISC-Vの機械語命令を実行するためには、1回以上のメモリアクセスが必要となりますが、
そのアクセスは「暗黙的」と「明示的」に分けられます。
各命令の実行時には、実行するコード化された命令を取得するために、
メモリリード(命令フェッチ)が行われます。
多くのRISC-V命令は、命令フェッチ以外のメモリアクセスを行いません。
特定のロード命令やストア命令は、命令によって決定されたアドレスに対して、
メモリのリードまたはライトを行います。
実行環境によっては、非特権ISAに記載されている以外のメモリアクセス(アドレス変換の実装など)を行うことがあります。

実行環境は、メモリアクセスの種類ごとに、空いていないアドレス空間のどの部分にアクセスできるかを決定します。
例えば、命令フェッチのために暗黙的に読める場所のセットは、
ロード命令で明示的に読める場所のセットと重なる部分があるかもしれませんし、
ストア命令で明示的に書ける場所のセットは、
読める場所のサブセットでしかないかもしれません。
通常、アクセスできないアドレスのメモリにアクセスしようとすると、
その命令に対して例外が発生します。
また、アドレス空間内の空いている場所には決してアクセスできません。

別段の定めがない限り、例外を発生させず、かつ副作用のない暗黙の読み出しは、
任意の時期に投機的に発生する可能性があり、
たとえその読み出しが必要であることをマシンが証明する前であってもです。 例えば、有効な実装では、最も早い機会にメインメモリのすべてを読み取り、
後の命令フェッチのためにフェッチ可能な(実行可能な)バイトをできるだけ多くキャッシュし、
命令フェッチのためにメインメモリを二度と読み取らないようにすることができます。 特定の暗黙のリードが、同じメモリロケーションへのライトの後にのみ順序付けられるようにするには、
ソフトウェアは、この目的のために定義された特定のフェンスまたは
キャッシュコントロール命令(第~\ref{chap:zifencei}章で定義されたFENCE.I命令など)
を実行する必要があります。

hartが行うメモリアクセス(暗黙的または明示的)は、
他のhartや同じメモリにアクセスできる他のエージェントが知覚すると、
異なる順序で行われているように見えることがあります。
しかし、このようなメモリアクセスの順序変更は、適用可能なメモリ一貫性モデルによって常に制約されます。
RISC-Vのデフォルトのメモリ一貫性モデルは、
RISC-V Weak Memory Ordering (RVWMO)で、第~\ref{ch:memorymodel}章とAppendicesで定義されています。
オプションとして、より強力なモデルであるTotal Store Orderingを
採用することもできます(第~\ref{sec:ztso}章で定義)。 
また、実行環境では、メモリアクセスの順序変更をさらに制限する
制約を加えることもできます。
RVWMOモデルは、RISC-Vの実装に許される最も弱いモデルであるため、
このモデル用に書かれたソフトウェアは、すべてのRISC-V実装の実際のメモリ一貫性ルールと互換性があります。
暗黙のリードと同様に、ソフトウェアは、想定されるメモリ一貫性モデルと実行環境の要件を超える
メモリアクセスの特定の順序を保証するために、フェンスまたはキャッシュ制御命令を実行する必要があります。

\begin{comment}
\section{Base Instruction-Length Encoding}

The base RISC-V ISA has fixed-length 32-bit instructions that must be
naturally aligned on 32-bit boundaries.  However, the standard RISC-V
encoding scheme is designed to support ISA extensions with
variable-length instructions, where each instruction can be any number
of 16-bit instruction {\em parcels} in length and parcels are
naturally aligned on 16-bit boundaries.  The standard compressed ISA
extension described in Chapter~\ref{compressed} reduces code size by
providing compressed 16-bit instructions and relaxes the alignment
constraints to allow all instructions (16 bit and 32 bit) to be
aligned on any 16-bit boundary to improve code density.

We use the term IALIGN (measured in bits) to refer to the instruction-address
alignment constraint the implementation enforces.  IALIGN is 32 bits in the
base ISA, but some ISA extensions, including the compressed ISA extension,
relax IALIGN to 16 bits.  IALIGN may not take on any value other than 16 or
32.

We use the term ILEN (measured in bits) to refer to the maximum
instruction length supported by an implementation, and which is always
a multiple of IALIGN.  For implementations supporting only a base
instruction set, ILEN is 32 bits.  Implementations supporting longer
instructions have larger values of ILEN.

Figure~\ref{instlengthcode} illustrates the standard RISC-V
instruction-length encoding convention.  All the 32-bit instructions
in the base ISA have their lowest two bits set to {\tt 11}.  The
optional compressed 16-bit instruction-set extensions have their
lowest two bits equal to {\tt 00}, {\tt 01}, or {\tt 10}.
\end{comment}

\section{ベース命令長エンコーディング}

ベースRISC-V ISAは、固定長の32ビット命令を備えており、
これらの命令は、32ビット境界で自然に整列する必要があります。
しかし、標準のRISC-Vエンコーディング方式は、可変長命令を持つISA拡張をサポートするように設計されており、
各命令は任意の数の16ビット命令{\em 区切り}になり、16ビット境界で自然に整列されます。
第~\ref{compressed}章で説明した標準の圧縮ISA拡張命令は、
圧縮された16ビット命令を提供することでコードサイズを縮小し、
すべての命令(16ビットと32ビット)を16ビット境界で整列させて
コード密度を向上させるための整列制約を緩和します。

ここでは、IALIGN(ビット単位)という用語を使用して、
実装が実施する命令-アドレスのアライメント制約を参照します。
IALIGNは、ベースISAでは32ビットですが、圧縮ISA拡張を含むいくつかのISA拡張では、
IALIGNを16ビットに緩和しています。
IALIGNは、16または32以外の値を取ることはできません。

ここでは、ある実装がサポートする最大の命令長を指すためにILEN(ビット単位)
という用語を使用し、これは常にIALIGNの倍数となります。
基本命令セットのみをサポートする実装では、ILENは32ビットです。
より長い命令をサポートする実装では、ILENの値が大きくなります。

図~\ref{instlengthcode}に、標準的なRISC-V命令長エンコード規則を示します。
ベースISAのすべての32ビット命令は、最下位の2ビットが{\tt 11}に固定されています。
オプションの圧縮された16ビット命令セット拡張は、
下位の2ビットが{\tt 00},{\tt 01}、または{\tt 10}のいずれかとなります。

\begin{comment}
\subsection*{Expanded Instruction-Length Encoding}

A portion of the 32-bit instruction-encoding space has been tentatively
allocated for instructions longer than 32 bits.  The entirety of this space is
reserved at this time, and the following proposal for encoding instructions
longer than 32 bits is not considered frozen.

Standard instruction-set extensions
encoded with more than 32 bits have additional low-order bits set to {\tt 1},
with the conventions for 48-bit and 64-bit lengths shown in
Figure~\ref{instlengthcode}.  Instruction lengths between 80 bits and 176 bits
are encoded using a 3-bit field in bits [14:12] giving the number of 16-bit
words in addition to the first 5$\times$16-bit words.  The encoding with bits
[14:12] set to {\tt 111} is reserved for future longer instruction encodings.


\begin{figure}[hbt]
{
\begin{center}
\begin{tabular}{ccccl}
\cline{4-4}
& & & \multicolumn{1}{|c|}{\tt xxxxxxxxxxxxxxaa} & 16-bit ({\tt aa}
$\neq$ {\tt 11})\\
\cline{4-4}
\\
\cline{3-4}
& & \multicolumn{1}{|c|}{\tt xxxxxxxxxxxxxxxx}
& \multicolumn{1}{c|}{\tt xxxxxxxxxxxbbb11} & 32-bit ({\tt bbb}
$\neq$ {\tt 111}) \\
\cline{3-4}
\\
\cline{2-4}
\hspace{0.1in}
& \multicolumn{1}{c|}{$\cdot\cdot\cdot${\tt xxxx} }
& \multicolumn{1}{c|}{\tt xxxxxxxxxxxxxxxx}
& \multicolumn{1}{c|}{\tt xxxxxxxxxx011111} & 48-bit \\
\cline{2-4}
\\
\cline{2-4}
\hspace{0.1in}
& \multicolumn{1}{c|}{$\cdot\cdot\cdot${\tt xxxx} }
& \multicolumn{1}{c|}{\tt xxxxxxxxxxxxxxxx}
& \multicolumn{1}{c|}{\tt xxxxxxxxx0111111} & 64-bit \\
\cline{2-4}
\\
\cline{2-4}
\hspace{0.1in}
& \multicolumn{1}{c|}{$\cdot\cdot\cdot${\tt xxxx} }
& \multicolumn{1}{c|}{\tt xxxxxxxxxxxxxxxx}
& \multicolumn{1}{c|}{\tt xnnnxxxxx1111111} & (80+16*{\tt nnn})-bit,
       {\tt nnn}$\neq${\tt 111} \\
\cline{2-4}
\\
\cline{2-4}
\hspace{0.1in}
& \multicolumn{1}{c|}{$\cdot\cdot\cdot${\tt xxxx} }
& \multicolumn{1}{c|}{\tt xxxxxxxxxxxxxxxx}
& \multicolumn{1}{c|}{\tt x111xxxxx1111111} & Reserved for $\geq$192-bits \\
\cline{2-4}
\\
Byte Address: & \multicolumn{1}{r}{base+4} & \multicolumn{1}{r}{base+2} & \multicolumn{1}{r}{base} & \\
 \end{tabular}
\end{center}
}
\caption{RISC-V instruction length encoding.  Only the 16-bit and 32-bit encodings are considered frozen at this time.}
\label{instlengthcode}
\end{figure}
\end{comment}

\subsection{命令長拡張エンコーディング}

32ビットの命令エンコード領域の一部は、32ビットよりも長い命令のために暫定的に割り当てられています。
現時点では、このスペースの全体が予約されており、
32ビットを超える命令をエンコードするための以下の提案は凍結されていないと考えられます。

32ビット以上でエンコードされた標準的な命令セット拡張は、
下位ビットが追加で1に設定されており、48ビットと64ビットの長さの規則は図~\ref{instlengthcode}に示されています。 
80ビットから176ビットまでの命令長は、
ビット[14:12]に16ビットワードの数を示す3ビットのフィールドを使用して符号化されますが、
これは最初の5$¥times$16ビットワードに加えて、16ビットワードの数を示すものです。
ビット[14:12]に{\tt 111}を設定したエンコーディングは、
将来のより長い命令のエンコーディングのために予約されています。

\begin{figure}[hbt]
{
\begin{center}
\begin{tabular}{ccccl}
\cline{4-4}
& & & \multicolumn{1}{|c|}{\tt xxxxxxxxxxxxxxaa} & 16-bit ({\tt aa}
$\neq$ {\tt 11})\\
\cline{4-4}
\\
\cline{3-4}
& & \multicolumn{1}{|c|}{\tt xxxxxxxxxxxxxxxx}
& \multicolumn{1}{c|}{\tt xxxxxxxxxxxbbb11} & 32-bit ({\tt bbb}
$\neq$ {\tt 111}) \\
\cline{3-4}
\\
\cline{2-4}
\hspace{0.1in}
& \multicolumn{1}{c|}{$\cdot\cdot\cdot${\tt xxxx} }
& \multicolumn{1}{c|}{\tt xxxxxxxxxxxxxxxx}
& \multicolumn{1}{c|}{\tt xxxxxxxxxx011111} & 48-bit \\
\cline{2-4}
\\
\cline{2-4}
\hspace{0.1in}
& \multicolumn{1}{c|}{$\cdot\cdot\cdot${\tt xxxx} }
& \multicolumn{1}{c|}{\tt xxxxxxxxxxxxxxxx}
& \multicolumn{1}{c|}{\tt xxxxxxxxx0111111} & 64-bit \\
\cline{2-4}
\\
\cline{2-4}
\hspace{0.1in}
& \multicolumn{1}{c|}{$\cdot\cdot\cdot${\tt xxxx} }
& \multicolumn{1}{c|}{\tt xxxxxxxxxxxxxxxx}
& \multicolumn{1}{c|}{\tt xnnnxxxxx1111111} & (80+16*{\tt nnn})-bit,
       {\tt nnn}$\neq${\tt 111} \\
\cline{2-4}
\\
\cline{2-4}
\hspace{0.1in}
& \multicolumn{1}{c|}{$\cdot\cdot\cdot${\tt xxxx} }
& \multicolumn{1}{c|}{\tt xxxxxxxxxxxxxxxx}
& \multicolumn{1}{c|}{\tt x111xxxxx1111111} & Reserved for $\geq$192-bits \\
\cline{2-4}
\\
Byte Address: & \multicolumn{1}{r}{base+4} & \multicolumn{1}{r}{base+2} & \multicolumn{1}{r}{base} & \\
 \end{tabular}
\end{center}
}
\caption{RISC-V命令長エンコーディング。現在では16ビットと32ビット命令長のエンコーディングのみが決定されている。}
\label{instlengthcode}
\end{figure}

\begin{comment}
\begin{commentary}
Given the code size and energy savings of a compressed format, we
wanted to build in support for a compressed format to the ISA encoding
scheme rather than adding this as an afterthought, but to allow
simpler implementations we didn't want to make the compressed format
mandatory. We also wanted to optionally allow longer instructions to
support experimentation and larger instruction-set extensions.
Although our encoding convention required a tighter encoding of the
core RISC-V ISA, this has several beneficial effects.

An implementation of the standard IMAFD ISA need only hold the
most-significant 30 bits in instruction caches (a 6.25\% saving).  On
instruction cache refills, any instructions encountered with either
low bit clear should be recoded into illegal 30-bit instructions
before storing in the cache to preserve illegal instruction exception
behavior.

Perhaps more importantly, by condensing our base ISA into a subset of
the 32-bit instruction word, we leave more space available for
non-standard and custom extensions.  In particular, the base RV32I ISA
uses less than 1/8 of the encoding space in the 32-bit instruction
word.  As described in Chapter~\ref{extensions}, an implementation
that does not require support for the standard compressed instruction
extension can map 3 additional non-conforming 30-bit instruction
spaces into the 32-bit fixed-width format, while preserving support
for standard $\geq$32-bit instruction-set extensions.  Further, if the
implementation also does not need instructions $>$32-bits in length,
it can recover a further four major opcodes for non-conforming extensions.
\end{commentary}
\end{comment}

\begin{commentary}
圧縮フォーマットのコードサイズとエネルギーの節約を考えると、
後付けではなく、ISAエンコーディングスキームに圧縮フォーマットのサポートを組み込みたかったのですが、
よりシンプルな実装を可能にするために、圧縮フォーマットを必須にはしませんでした。
また、実験やより大きな命令セットの拡張をサポートするために、
より長い命令をオプションで許可したいと考えました。
今回のエンコード方式では、RISC-V ISAのコア部分をよりタイトにエンコードする必要がありましたが、
これにはいくつかの利点があります。

標準的なIMAFD ISAの実装では、
命令キャッシュには上位の30ビットを保持するだけで良いです(6.25\%の削減)。 命令キャッシュのリフィルでは、不正な命令例外の動作を維持するために、
いずれかの下位ビットがクリアされている命令は、
キャッシュに格納する前に不正な30ビット命令に再コードされるべきです。

さらに重要なことは、ベースISAを32ビット命令語のサブセットに凝縮することで、
非標準やカスタムの拡張に利用できるスペースを確保していることです。
特に、ベースとなるRV32I ISAでは、32ビット命令語のエンコーディングスペースの1/8以下しか使用していません。
第~\ref{extensions}章で説明したように、標準的な圧縮命令拡張のサポートを必要としない実装では、
標準的な$\geq$32ビット命令セット拡張のサポートを維持しながら、
さらに3つの標準ではない30ビット命令スペースを32ビット固定幅フォーマットにマッピングすることができます。
さらに、32ビットよりも長い命令を必要としない実装であれば、
さらに4つの主要なオペコードを標準ではない拡張に対応させることができます。
\end{commentary}

\begin{comment}
Encodings with bits [15:0] all zeros are defined as illegal
instructions.  These instructions are considered to be of minimal
length: 16 bits if any 16-bit instruction-set extension is present,
otherwise 32 bits.  The encoding with bits [ILEN-1:0] all ones is also
illegal; this instruction is considered to be ILEN bits long.
\end{comment}

ビット[15:0]がすべてゼロのエンコーディングは、不正な命令として定義されます。 
これらの命令は、16ビットの命令セット拡張がある場合は16ビット、それ以外の場合は32ビットという最小の長さとみなされます。
また、[ILEN-1:0]ビットがすべて1のエンコーディングも不正な命令で、この命令はILENビット長とみなされます。

\begin{comment}
\begin{commentary}
We consider it a feature that any length of instruction containing all
zero bits is not legal, as this quickly traps erroneous jumps into
zeroed memory regions. Similarly, we also reserve the instruction
encoding containing all ones to be an illegal instruction, to catch
the other common pattern observed with unprogrammed non-volatile
memory devices, disconnected memory buses, or broken memory devices.

Software can rely on a naturally aligned 32-bit word containing zero to
act as an illegal instruction on all RISC-V implementations, to be used
by software where an illegal instruction is explicitly desired.
Defining a corresponding known illegal value for all ones is more
difficult due to the variable-length encoding.  Software cannot
generally use the illegal value of ILEN bits of all 1s, as software
might not know ILEN for the eventual target machine (e.g., if software
is compiled into a standard binary library used by many different
machines).  Defining a 32-bit word of all ones as illegal was also
considered, as all machines must support a 32-bit instruction size, but
this requires the instruction-fetch unit on machines with ILEN$>$32
report an illegal instruction exception rather than an access-fault
exception when such an instruction borders a protection boundary,
complicating variable-instruction-length fetch and decode.
\end{commentary}
\end{comment}

\begin{commentary}
すべての長さの命令において、すべてのビットがゼロの命令は、
ゼロのメモリ領域への誤ったジャンプをすぐに捕捉するため、
不正な命令となっていることを特徴としています。
同様に、プログラムされていない不揮発性メモリデバイス、
切断されたメモリバス、または壊れたメモリデバイスで観察されるもう一つの一般的な
パターンを捕らえるために、すべての1を含む命令エンコーディングも不正な命令として予約します。

ソフトウェアは、すべてのRISC-Vの実装において、
0を含む自然にアラインされた32ビットのワードが不正な命令として機能することに依存し、
不正な命令が明示的に望まれるソフトウェアによって使用されます。
すべての1に対応する既知のイリーガル値を定義することは、可変長エンコーディングのため、より困難です。
ソフトウェアは、最終的なターゲット・マシンのILENを知らない可能性があるため
(例えば、ソフトウェアが多くの異なるマシンで使用される標準的なバイナリ・ライブラリにコンパイルされている場合)、
すべての1のILENビットのイリーガル値を一般的に使用することはできません。 
すべてのマシンが32ビットの命令サイズをサポートする必要があるため、
すべての1からなる32ビットのワードを不正命令と定義することも検討されましたが、
この場合、ILEN$>$32を持つマシンの命令フェッチ・ユニットは、
そのような命令が保護境界を越えたときに、アクセス・フォールト例外ではなく不正命令例外を報告する必要があり、
可変命令長のフェッチおよびデコードが複雑になります。
\end{commentary}

\begin{comment}
RISC-V base ISAs have either little-endian or big-endian memory systems,
with the privileged architecture further defining bi-endian operation.
Instructions are stored in memory as a sequence of 16-bit little-endian
parcels, regardless of memory system endianness.
Parcels forming one instruction are stored at increasing
halfword addresses, with the lowest-addressed parcel holding the
lowest-numbered bits in the instruction specification.
\end{comment}

RISC-Vの基本ISAは、リトルエンディアンまたはビッグエンディアンのメモリシステムを持ち、
特権アーキテクチャではさらにバイエンディアン動作が定義されています。
命令は、メモリシステムのエンディアンに関係なく、
16ビットのリトルエンディアン区画のシーケンスとしてメモリに格納されます。
1つの命令を構成する区画は、ハーフワードアドレスの増加に伴って格納され、
最も低いアドレスの区画には、命令仕様の最下位ビットが格納されています。

\begin{comment}
\begin{commentary}
We originally chose little-endian byte ordering for the RISC-V memory system
because little-endian systems are currently dominant commercially (all
x86 systems; iOS, Android, and Windows for ARM).  A minor point is
that we have also found little-endian memory systems to be more
natural for hardware designers.  However, certain application areas,
such as IP networking, operate on big-endian data structures, and
certain legacy code bases have been built assuming big-endian
processors, so we have defined big-endian and bi-endian variants of RISC-V.

We have to fix the order in which instruction parcels are stored in
memory, independent of memory system endianness, to ensure that the
length-encoding bits always appear first in halfword address
order. This allows the length of a variable-length instruction to be
quickly determined by an instruction-fetch unit by examining only the
first few bits of the first 16-bit instruction parcel.

We further make the instruction parcels themselves little-endian to decouple
the instruction encoding from the memory system endianness altogether.
This design benefits both software tooling and bi-endian hardware.
Otherwise, for instance, a RISC-V assembler or disassembler would always need
to know the intended active endianness, despite that in bi-endian systems, the
endianness mode might change dynamically during execution.
In contrast, by giving instructions a fixed endianness, it is sometimes
possible for carefully written software to be endianness-agnostic even in
binary form, much like position-independent code.

The choice to have instructions be only little-endian does have consequences,
however, for RISC-V software that encodes or decodes machine instructions.
Big-endian JIT compilers, for example, must swap the byte order when storing
to instruction memory.

Once we had decided to fix on a little-endian instruction encoding, this
naturally led to placing the length-encoding bits in the LSB positions of the
instruction format to avoid breaking up opcode fields.
\end{commentary}
\end{comment}

\begin{commentary}
RISC-Vのメモリシステムにリトルエンディアン方式のバイトオーダを採用したのは、
現在、商業的にリトルエンディアン方式が主流となっているからです
(x86系のすべてのシステム、ARM系のiOS、Android、Windowsなど)。 
また、ハードウェア設計者にとっても、リトルエンディアンのメモリシステムの方が自然であると考えたからです。 
しかし、IPネットワークなどの特定のアプリケーション分野では、ビッグエンディアンのデータ構造で動作しますし、
レガシーのコードベースもビッグエンディアンのプロセッサを想定して作られていますので、
RISC-Vにビッグエンディアンとバイエンディアンのバリエーションを定義しました。

メモリシステムのエンディアンとは無関係に、命令区画がメモリに格納される順序を固定し、
長さをエンコードするビットがハーフワードアドレス順に常に最初に現れるようにしなければなりません。
これにより、命令フェッチユニットは、最初の16ビットの命令区画の最初の数ビットを調べるだけで、
可変長の命令の長さをすばやく判断することができます。

さらに、命令区画自体をリトルエンディアンにすることで、
命令のエンコーディングとメモリシステムのエンディアンを完全に切り離しています。
この設計は、ソフトウェアツールとバイエンディアンハードウェアの両方にメリットがあります。
例えば、RISC-Vのアセンブラやディスアセンブラは、
実行中にエンディアンモードが動的に変化する可能性があるにもかかわらず、
常に意図したアクティブなエンディアンを知る必要があります。
これに対して、命令に固定のエンディアンネスを与えることで、
注意深く書かれたソフトウェアは、位置に依存しないコードのように、
バイナリ形式であってもエンディアンネスに依存しないことができる場合があります。

しかし、リトルエンディアンのみを選択したことは、
機械語命令をエンコードまたはデコードするRISC-Vソフトウェアに影響を与えます。
例えば、ビッグエンディアンのJITコンパイラは、
命令メモリに格納する際にバイト順を入れ替える必要があります。

一度命令のエンコーディングをリトルエンディアンに固定することが決めると、
当然、長さエンコーディングビットを命令フォーマットのLSB位置に配置して、
オペコードフィールドを分割しないようにしています。
\end{commentary}

\begin{comment}
\section{Exceptions, Traps, and Interrupts}
\label{sec:trap-defn}

We use the term {\em exception} to refer to an unusual condition
occurring at run time associated with an instruction in the current
RISC-V hart.  We use the term {\em interrupt} to refer to an external
asynchronous event that may cause a RISC-V hart to experience an
unexpected transfer of control.  We use the term {\em trap} to refer
to the transfer of control to a trap handler caused by either an
exception or an interrupt.

The instruction descriptions in following chapters describe conditions
that can raise an exception during execution.  The general behavior of
most RISC-V EEIs is that a trap to some handler occurs when an
exception is signaled on an instruction (except for floating-point
exceptions, which, in the standard floating-point extensions, do not
cause traps).  The manner in which interrupts are generated, routed
to, and enabled by a hart depends on the EEI.
\end{comment}

\section{例外、トラップ、割り込み}
\label{sec:trap-defn}

{\em 例外(exception)}とは、現在のRISC-V hartの命令に関連してランタイムに発生する異常な状態を指します。
外部からの非同期イベントで、RISC-V hartに予期せぬ制御権の移譲が発生することを指しています。
また、例外や割り込みによるトラップハンドラへの制御の移行については、
{\em トラップ(trap)}という用語を使用しています。

以下の章の命令説明では、実行中に例外が発生する条件を説明しています。 
ほとんどのRISC-V EEIの一般的な動作は、
命令で例外がシグナリングされると、何らかのハンドラへのトラップが発生します
(標準の浮動小数点拡張機能ではトラップが発生しない浮動小数点例外を除く)。 
割り込みがどのように生成され、どのようにルーティングされ、どのように有効化されるのかは、EEIによって異なります。

\begin{comment}
\begin{commentary}
Our use of ``exception'' and ``trap'' is compatible with that in the IEEE-754
floating-point standard.
\end{commentary}
\end{comment}

\begin{commentary}
私たちが用いる``例外(exception)''と``トラップ(trap)''は、
IEEE-754浮動小数点規格のものと互換性があります。
\end{commentary}

\begin{comment}
How traps are handled and made visible to software running on the hart
depends on the enclosing execution environment.  From the perspective
of software running inside an execution environment, traps encountered
by a hart at runtime can have four different effects:
\begin{description}
  \item[Contained Trap:] The trap is visible to, and handled by,
    software running inside the execution environment.  For example,
    in an EEI providing both supervisor and user
    mode on harts, an ECALL by a user-mode hart will generally result
    in a transfer of control to a supervisor-mode handler running on
    the same hart.  Similarly, in the same environment, when a hart is
    interrupted, an interrupt handler will be run in supervisor mode
    on the hart.
  \item[Requested Trap:] The trap is a synchronous exception that is
    an explicit call to the execution environment requesting an action
    on behalf of software inside the execution environment.  An
    example is a system call.  In this case, execution may or may not
    resume on the hart after the requested action is taken by the
    execution environment.  For example, a system call could remove the
    hart or cause an orderly termination of the entire execution environment.
  \item[Invisible Trap:] The trap is handled transparently by the
    execution environment and execution resumes normally after the
    trap is handled.  Examples include emulating missing instructions,
    handling non-resident page faults in a demand-paged virtual-memory
    system, or handling device interrupts for a different job in a
    multiprogrammed machine.  In these cases, the software running
    inside the execution environment is not aware of the trap (we
    ignore timing effects in these definitions).
  \item[Fatal Trap:] The trap represents a fatal failure and causes
    the execution environment to terminate execution.  Examples
    include failing a virtual-memory page-protection check or allowing
    a watchdog timer to expire.  Each EEI should define how execution
    is terminated and reported to an external environment.
\end{description}
\end{comment}

トラップがどのように処理され、hart上で動作するソフトウェアに表示されるかは、
周囲の実行環境によって異なります。
実行環境の中で動作するソフトウェアの視点から見ると、
実行時にhartが遭遇するトラップは4つの異なる効果をもたらします。

\begin{description}
  \item [Contained Trap:] このトラップは、実行環境内で動作するソフトウェアから見え、処理されます。
  例えば、スーパーバイザーモードとユーザーモードの両方のhartを提供しているEEIでは、
  ユーザーモードのhartによるECALLは、一般的に、同じhart上で実行されている
  スーパーバイザーモードのハンドラに制御を移すことになります。
  同様に、同じ環境下では、hartが割り込みを受けると、
  そのhart上でスーパバイザモードの割り込みハンドラが実行されます。
  \item[Requested Trap:] このトラップでは、実行環境に明示的な呼び出しを行い、
  実行環境内のソフトウェアに代わってアクションを要求する同期例外のことです。
  例として、システムコールがあります。
  この場合、要求されたアクションが実行環境で実行された後、hartで実行が再開される場合もあれば、
  再開されない場合もあります。
  例えば、システムコールは、hartを削除したり、
  実行環境全体の秩序ある終了を引き起こす可能性があります。
  \item[Invisible Trap:] このトラップは実行環境によって透過的に処理され、
  トラップの処理後に実行が正常に再開されます。
  例えば、欠落した命令のエミュレーション、デマンド・ページング方式の
  仮想メモリ・システムにおける非常駐ページ・フォールトの処理、
  マルチプログラム・マシンにおける別のジョブのためのデバイス割込みの処理などがあります。
  これらのケースでは、実行環境内で動作しているソフトウェアは
  トラップを意識しません(ここではタイミングの影響を無視しています)。
  \item[Fatal Trap:] このトラップは致命的な失敗を表し、実行環境が実行を終了させます。
  例としては、仮想メモリのページ保護チェックの失敗や、
  ウォッチドッグ・タイマーの期限切れなどがあります。
  各EEIは、どのように実行を終了し、外部環境に報告するかを定義する必要があります。
\end{description}

\begin{comment}
Table~\ref{table:trapcharacteristics} shows the characteristics of each
kind of trap.

\begin{table}[hbt]
  \centering
  \begin{tabular}{|l|c|c|c|c|}
      \hline
      & Contained & Requested & Invisible & Fatal\\
      \hline
      Execution terminates   & No & No$^{1}$ & No  & Yes \\
      Software is oblivious  & No & No       & Yes & Yes$^{2}$ \\
      Handled by environment & No & Yes      & Yes & Yes \\
      \hline
  \end{tabular}
  \caption{Characteristics of traps. Notes: 1) Termination may be
    requested. 2) Imprecise fatal traps might be observable by software.}
\label{table:trapcharacteristics}
\end{table}
\end{comment}

表~\ref{table:trapcharacteristics}に各トラップの特徴を示します。

\begin{table}[hbt]
  \centering
  \begin{tabular}{|l|c|c|c|c|}
      \hline
      & Contained & Requested & Invisible & Fatal\\
      \hline
      実行終了             & No & No$^{1}$ & No  & Yes \\
      ソフトウェアから観測できる  & No & No       & Yes & Yes$^{2}$ \\
      実行環境により処理される & No & Yes      & Yes & Yes \\
      \hline
  \end{tabular}
  \caption{トラップの特徴. 注: 1) 実行の終了がリクエストされる. 2) 不正確なfatal trapがソフトウェアから観測可能.}
\label{table:trapcharacteristics}
\end{table}

\begin{comment}
The EEI defines for each trap whether it is handled precisely, though
the recommendation is to maintain preciseness where possible.
Contained and requested traps can be observed to be imprecise by
software inside the execution environment.  Invisible traps, by
definition, cannot be observed to be precise or imprecise by software
running inside the execution environment.  Fatal traps can be observed
to be imprecise by software running inside the execution environment,
if known-errorful instructions do not cause immediate termination.

Because this document describes unprivileged instructions, traps are
rarely mentioned.  Architectural means to handle contained traps are
defined in the privileged architecture manual, along with other
features to support richer EEIs.  Unprivileged instructions that are
defined solely to cause requested traps are documented here.
Invisible traps are, by their nature, out of scope for this document.
Instruction encodings that are not defined here and not defined by
some other means may cause a fatal trap.
\end{comment}

EEIでは、各トラップが正確に処理されるかどうかを定義していますが、
可能な限り正確さを維持することを推奨しています。
Contained TrapとRequested Trapは、実行環境内のソフトウェアによって不正確であることが確認できます。
Invisible Trapは、定義上、実行環境内で実行されているソフトウェアによって
正確であるか不正確であるかを観察することはできません。 Fatal Trapは、既知のエラー命令が即時終了を引き起こさない場合、
実行環境内で動作するソフトウェアによって不正確であることが確認できます。

このドキュメントでは、非特権的な命令について説明しているため、
トラップについてはほとんど触れられていません。 
含まれるトラップを処理するアーキテクチャ手段は、よりリッチなEEIをサポートする他の機能とともに、
特権アーキテクチャ・マニュアルで定義されています。 
ここでは、Requested Trapを引き起こすためだけに定義された非特権命令について説明します。
Invisible Trapは、その性質上、このドキュメントの対象外となります。
ここで定義されておらず、他の手段でも定義されていない命令エンコーディングは、
致命的なトラップを引き起こす可能性があります。

\begin{comment}
\section{UNSPECIFIED Behaviors and Values}

The architecture fully describes what implementations must do and any
constraints on what they may do.  In cases where the architecture
intentionally does not constrain implementations, the term \unspecified\
is explicitly used.

The term \unspecified\ refers to a behavior or value that is
intentionally unconstrained.  The definition of these behaviors or
values is open to extensions, platform standards, or implementations.
Extensions, platform standards, or implementation documentation may
provide normative content to further constrain cases that the base
architecture defines as \unspecified.

Like the base architecture, extensions should fully describe allowable
behavior and values and use the term \unspecified\ for cases that are
intentionally unconstrained.  These cases may be constrained or defined
by other extensions, platform standards, or implementations.
\end{comment}

\section{``定義されていない''動作と値}

アーキテクチャは、実装が何をしなければならないか、何をしてもよいかという制約を完全に記述しています。
ーキテクチャが意図的に実装を制約しない場合には、 ``定義されていない(unspecified)'' という用語が明示的に使用されます。

意図的に制約されていない動作や値を意味します。
これらの動作や値の定義は、拡張機能、プラットフォーム標準、
または実装に開放されています。
拡張機能、プラットフォーム標準、または実装文書は、基本アーキテクチャが ``定義されていない'' 
と定義したケースをさらに制約するための規範的な内容を提供することができます。

基本アーキテクチャと同様に、拡張機能は許容される動作と値を完全に記述する必要があり、
意図的に制約を受けないケースには ``定義されていない(unspecified)'' という用語を使用します。 
これらのケースは、他の拡張機能、プラットフォームの標準、
または実装によって制約されたり定義されたりします。
