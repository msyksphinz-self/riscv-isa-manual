\begin{comment}
\chapter{RV32E Base Integer Instruction Set, Version 1.9}
\end{comment}
\chapter{RV32E ベース整数命令セット, Version 1.9}
\label{rv32e}

\begin{comment}
This chapter describes a draft proposal for the RV32E base integer
instruction set, which is a reduced version of RV32I designed for
embedded systems.  The only change is to reduce the number of integer
registers to 16.  This chapter only outlines the differences between
RV32E and RV32I, and so should be read after Chapter~\ref{rv32}.
\end{comment}

この章では、組み込みシステム用に設計されたRV32Iの縮小版である、
RV32Eベース整数命令セットのドラフト案について説明します。
唯一の変更点は、整数レジスタの数を16に減らすことです。
本章では、RV32EとRV32Iの違いについてのみ説明していますので、
第~\ref{rv32}章の後にお読みください。

\begin{commentary}
\begin{comment}
RV32E was designed to provide an even smaller base core for embedded
microcontrollers.  Although we had mentioned this possibility in
version 2.0 of this document, we initially resisted defining this
subset. However, given the demand for the smallest possible 32-bit
microcontroller, and in the interests of preempting fragmentation in
this space, we have now defined RV32E as a fourth standard base ISA in
addition to RV32I, RV64I, and RV128I.  There is also interest in
defining an RV64E to reduce context state for highly threaded 64-bit
processors.
\end{comment}

RV32Eは、組み込みマイクロコントローラ用に、より小さなベースコアを提供するために設計されました。 
本ドキュメント2.0版ではその可能性に言及していましたが、
当初はこのサブセットを定義することに抵抗がありました。
しかし、可能な限り小さい32ビットマイクロコントローラが求められていることや、
この分野の断片化を防ぐために、RV32I、RV64I、RV128Iに加えて、
RV32Eを4つ目の標準ベースISAとして定義しました。
また、非常にスレッド化された64ビットプロセッサのコンテキストステートを
軽減するために、RV64Eを定義することにも関心があります。
\end{commentary}

\begin{comment}
\section{RV32E Programmers' Model}
\end{comment}
\section{RV32E プログラマーモデル}

\begin{comment}
RV32E reduces the integer register count to 16 general-purpose
registers, ({\tt x0}--{\tt x15}), where {\tt x0} is a dedicated zero
register.
\end{comment}

RV32Eでは、整数レジスタを16本の汎用レジスタ({\tt x0}--{\tt x15})に削減しており、{\tt x0}は専用のゼロレジスタです。

\begin{commentary}
\begin{comment}
We have found that in the small RV32I core designs, the upper 16
registers consume around one quarter of the total area of the core
excluding memories, thus their removal saves around 25\% core area
with a corresponding core power reduction.
\end{comment}

小型のRV32Iコアでは、上位16本のレジスタがメモリを除くコアの総面積の約4分の1を占めており、
これを取り除くことで約25\%のコア面積を削減し、それに伴ってコア電力も削減できることがわかりました。
\end{commentary}

\begin{commentary}
\begin{comment}
This change requires a different calling convention and ABI.  In
particular, RV32E is only used with a soft-float calling convention.
A new embedded ABI is under consideration that would work across RV32E
and RV32I.
\end{comment}
この変更には、異なる呼び出し規則とABIが必要です。
特に、RV32Eはsoft-floatの呼び出し規則でのみ使用されます。
現在、RV32EとRV32Iの間で動作する新しい組み込みABIを検討中です。
\end{commentary}

\begin{comment}
\section{RV32E Instruction Set}
\end{comment}
\section{RV32E命令セット}

\begin{comment}
RV32E uses the same instruction-set encoding as RV32I, except that
only registers {\tt x0}--{\tt x15} are provided.  Any future standard
extensions will not make use of the instruction bits freed up by the
reduced register-specifier fields and so these are designated for
custom extensions.
\end{comment}

RV32Eは、RV32Iと同じ命令セットエンコーディングを採用していますが、
レジスタが{\tt x0}~{\tt x15}のみとなっています。
将来の標準的な拡張機能では、レジスタ指定フィールドの減少によって解放される
命令ビットを使用しないため、これらはカスタム拡張用に指定されています。

\begin{commentary}
\begin{comment}
RV32E can be combined with all current standard extensions. Defining the F, D,
and Q extensions as having a 16-entry floating point register file when
combined with RV32E was considered but decided against. To support
systems with reduced floating-point register state, we intend to
define a ``Zfinx'' extension that makes floating-point computations use the
integer registers, removing the floating-point loads, stores, and moves between
floating point and integer registers.
\end{comment}

RV32Eは現在のすべての標準的な拡張機能と組み合わせることができます。
なお、F、D、Qの各拡張機能をRV32Eと組み合わせて16エントリの浮動小数点レジスタファイルを持つように
定義することも検討されましたが、断念しました。
浮動小数点レジスタの状態が減少したシステムをサポートするために、
浮動小数点演算に整数レジスタを使用し、
浮動小数点のロード、ストア、浮動小数点と整数レジスタ間の移動を削除する ``Zfinx'' 拡張機能を定義する予定です。
\end{commentary}
