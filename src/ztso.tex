\begin{comment}
\chapter{``Ztso'' Standard Extension for Total Store Ordering, v0.1}
\end{comment}
\chapter{トータルストアオーダリングのための``Ztso''標準拡張, v0.1}
\label{sec:ztso}

\begin{comment}
This chapter defines the ``Ztso'' extension for the RISC-V Total Store Ordering (RVTSO) memory consistency model.
RVTSO is defined as a delta from RVWMO, which is defined in Chapter~\ref{sec:rvwmo}.
\end{comment}
本章では,RISC-Vトータルストアオーダリング(RISC-V Total Store Ordering: RVTSO) のメモリ一貫性モデルの拡張機能である ``Ztso'' を定義します。
本章では、RVTSOの拡張機能であるZtsoを定義します。RVTSOは、~\ref{sec:rvwmo}節で定義されているRVWMOからの差分として定義されます。

\begin{commentary}
  \begin{comment}
  The Ztso extension is meant to facilitate the porting of code originally written for the x86 or SPARC architectures, both of which use TSO by default.
  It also supports implementations which inherently provide RVTSO behavior and want to expose that fact to software.
  \end{comment}
  Ztso拡張は、デフォルトでTSOを使用しているx86やSPARCアーキテクチャ向けに書かれたコードの移植を容易にするためのものです。
  また、本質的にRVTSOの動作を提供し、その事実をソフトウェアに公開したいと考えている実装にも対応しています。
\end{commentary}

\begin{comment}
RVTSO makes the following adjustments to RVWMO:
\end{comment}
RVTSOはRVWMOに対して以下の調整を行っています:

\begin{comment}
\begin{itemize}
  \item All load operations behave as if they have an acquire-RCpc annotation
  \item All store operations behave as if they have a release-RCpc annotation.
  \item All AMOs behave as if they have both acquire-RCsc and release-RCsc annotations.
\end{itemize}
\end{comment}
\begin{itemize}
\item すべてのロード操作は、acquire-RCpcアノテーションを持っているかのように動作します。
\item すべてのストア操作はrelease-RCpcアノテーションを持っているかのように動作します。
\item すべてのAMOはacquire-RCscとrelease-RCscの両方のアノテーションを持っているかのように動作します。
\end{itemize}

\begin{commentary}
  \begin{comment}
  These rules render all PPO rules except \ref{ppo:fence}--\ref{ppo:rcsc} redundant.
  They also make redundant any non-I/O fences that do not have both PW and SR set.
  Finally, they also imply that no memory operation will be reordered past an AMO in either direction.

  In the context of RVTSO, as is the case for RVWMO, the storage ordering annotations are concisely and completely defined by PPO rules \ref{ppo:acquire}--\ref{ppo:rcsc}. In both of these memory models, it is the \nameref{rvwmo:ax:load} that allows a hart to forward a value from its store buffer to a subsequent (in program order) load---that is to say that stores can be forwarded locally before they are visible to other harts.
  \end{comment}

  これらのルールは、PPOルールのうち、\ref{ppo:fence}--\ref{ppo:rcsc}以外のすべてのルールを冗長にします。
  また、PWとSRの両方が設定されていない非I/O fenceも冗長になります。
  最後に、これらのルールは、AMOを過ぎてどちらかの方向に並び替えられるメモリ操作がないことを意味します。

  RVTSOの場合も、RVWMOの場合と同様に、ストレージの順序付けのアノテーションは、PPOルール\ref{ppo:acquire}--\ref{ppo:rcsc}によって簡潔かつ完全に定義されています。
  どちらのメモリモデルでも、hartがストアバッファから後続のロード(プログラム順)に値を転送することを可能にするのは、\nameref{rvwmo:ax:load}です---つまり、ストアが他のhartから見える前に,ローカルに転送することができるのです。


\end{commentary}

\begin{comment}
In spite of the fact that Ztso adds no new instructions to the ISA, code written assuming RVTSO will not run correctly on implementations not supporting Ztso.
Binaries compiled to run only under Ztso should indicate as such via a flag in the binary, so that platforms which do not implement Ztso can simply refuse to run them.
\end{comment}

ZtsoはISAに新しい命令を追加しないにもかかわらず、RVTSOを想定して書かれたコードは、Ztsoをサポートしていない実装では正しく動作しません。
Ztsoでのみ動作するようにコンパイルされたバイナリは、バイナリ内のフラグによってそのように示されるべきであり、Ztsoを実装していないプラットフォームは単純に実行を拒否することができます。
